% Template for ICASSP-2021 paper; to be used with:
%          spconf.sty  - ICASSP/ICIP LaTeX style file, and
%          IEEEbib.bst - IEEE bibliography style file.
% --------------------------------------------------------------------------
\documentclass{article}
\usepackage{../icip/spconf}
\usepackage{amsmath,graphicx}

% comentar para versão final======
\onecolumn
%=================================
% Example definitions.
% --------------------
\def\x{{\mathbf x}}
\def\L{{\cal L}}

% Title.
% ------
\title{Improvement in the estimation of point of regard for low-cost eye-trackers by using visual attention models}
%
% Single address.
% ---------------
\name{Ronaldo de Freitas Zampolo\thanks{Thanks to XYZ agency for funding.}}
\address{Federal University of Par\'a -- UFPA\\
         Department of Computer and Telecom Engineering -- FCT\\
         Signal Processing Laboratory -- LaPS\\
         %Institute of Technology\\
	 66075-110  Bel\'em-PA, Brazil}
%
% For example:
% ------------
%\address{School\\
%	Department\\
%	Address}
%
% Two addresses (uncomment and modify for two-address case).
% ----------------------------------------------------------
%\twoauthors
%  {A. Author-one, B. Author-two\sthanks{Thanks to XYZ agency for funding.}}
%	{School A-B\\
%	Department A-B\\
%	Address A-B}
%  {C. Author-three, D. Author-four\sthanks{The fourth author performed the work
%	while at ...}}
%	{School C-D\\
%	Department C-D\\
%	Address C-D}
%
\begin{document}
%\ninept
%
\maketitle
%
\begin{abstract}
	% context
	Researchers interested in studying eye movement were the first to use eye-trackers in their investigations. Psychologists came next, mostly trying to find out connextions between eye-movements and cognitive states. Currently, many areas benefit from the tool, like neuromarketing, design, digital gaming, visual attention, and emotion modeling. Now, eye-tracking systems are widely spread and there is a large variety of models, prices and quality. The choice of a specific equipment depends on the requirements of the intended application. 

	% gap
	Crowdsoucing is another area which would be enormously boosted with the aid of such a tool. However, for crowdsourcing to really work, the data acquisition apparatus, including the eye-tracker shoud be as cheap as possible. Particularly, point of regard accuracy is an issue. Generally, low cost eye-trackers suffer from poor quality, because the apparatus to acquire the user image is simple, as expected, based on webcams without infra-red illumination.
	
	% proposal
	This paper presents a technique to improve the accuracy of low-cost eye-trackers by using dynamic visual attention models. Basically, the visual attention model works as a filter to reduce noise in point of regard estimations and at the same time correct them to the most probable scene location.
	
	% methodology
	The technique efficiency is demonstrated experimentaly where we use two eye-trackers in parallel to collect data: one is a comercial equipment with pretty good accuracy (acc here) whereas the other is a low-cost webcam-based software eye-tracker (measured acc of...). We performed comparisons between original and corrected low-cost PoRs against comercial PoR by using heatmaps and scanpath metrics.

	% results and conclusions
	Our results show an improvement... 


\end{abstract}
%
\begin{keywords}
	Eye-tracking, low-cost eye-tracker, visual attention model, crowdsourcing, point of regard estimation.
%One, two, three, four, five
\end{keywords}
%
\section{Introduction}
\label{sec:intro}


% Below is an example of how to insert images. Delete the ``\vspace'' line,
% uncomment the preceding line ``\centerline...'' and replace ``imageX.ps''
% with a suitable PostScript file name.
% -------------------------------------------------------------------------
\begin{figure}[htb]

\begin{minipage}[b]{1.0\linewidth}
  \centering
%  \centerline{\includegraphics[width=8.5cm]{image1}}
%  \vspace{2.0cm}
  \centerline{(a) Result 1}\medskip
\end{minipage}
%
\begin{minipage}[b]{.48\linewidth}
  \centering
%  \centerline{\includegraphics[width=4.0cm]{image3}}
%  \vspace{1.5cm}
  \centerline{(b) Results 3}\medskip
\end{minipage}
\hfill
\begin{minipage}[b]{0.48\linewidth}
  \centering
%  \centerline{\includegraphics[width=4.0cm]{image4}}
%  \vspace{1.5cm}
  \centerline{(c) Result 4}\medskip
\end{minipage}
%
\caption{Example of placing a figure with experimental results.}
\label{fig:res}
%
\end{figure}


% To start a new column (but not a new page) and help balance the last-page
% column length use \vfill\pagebreak.
% -------------------------------------------------------------------------
%\vfill
%\pagebreak

\vfill\pagebreak

\section{REFERENCES}
\label{sec:refs}
% References should be produced using the bibtex program from suitable
% BiBTeX files (here: strings, refs, manuals). The IEEEbib.bst bibliography
% style file from IEEE produces unsorted bibliography list.
% -------------------------------------------------------------------------
\bibliographystyle{../icip/IEEEbib}
%\bibliography{strings,refs}

\end{document}
